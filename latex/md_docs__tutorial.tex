Below are examples illustrating a few common use-\/cases.\hypertarget{md_docs__tutorial_Setup}{}\section{Setup}\label{md_docs__tutorial_Setup}
{\itshape To be completed...}\hypertarget{md_docs__tutorial_Examples}{}\section{Examples}\label{md_docs__tutorial_Examples}
\hypertarget{md_docs__tutorial_JSON}{}\subsection{Load / Write J\+S\+ON}\label{md_docs__tutorial_JSON}
\href{https://en.wikipedia.org/wiki/JSON}{\tt J\+S\+ON} is a great format for configuring applications and for exchanging data over the internet\+:

Assuming an {\ttfamily input.\+json} file as follows\+:


\begin{DoxyCode}
\{
    "Key 1": "Hello World!"
\}
\end{DoxyCode}


the following code\+:


\begin{DoxyCode}
\textcolor{comment}{// Setup JSON file}

File jsonFile = File::Path(Path::String(\textcolor{stringliteral}{"/path/to/input.json"})) ;

\textcolor{comment}{// Create object by loading JSON file}

Object \textcolor{keywordtype}{object} = Object::Load(jsonFile) ;

\textcolor{comment}{// Get object value}

String value = \textcolor{keywordtype}{object}[\textcolor{stringliteral}{"Key 1"}].getString() ; \textcolor{comment}{// "Hello World!"}

\textcolor{comment}{// Set object value}

\textcolor{keywordtype}{object}[\textcolor{stringliteral}{"Key 2"}] = Object::Integer(123) ;

\textcolor{comment}{// Save object}

\textcolor{keywordtype}{object}.writeToFile(File::Path(Path::String(\textcolor{stringliteral}{"/path/to/output.json"}))) ;
\end{DoxyCode}


will return an {\ttfamily output.\+json} file as\+:


\begin{DoxyCode}
\{
    "Key 1": "Hello World!",
    "Key 2": 123
\}
\end{DoxyCode}
 